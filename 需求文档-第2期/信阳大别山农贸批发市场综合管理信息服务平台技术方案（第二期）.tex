\documentclass[12pt,a4paper,openany]{ctexrep}
%ctexrep:报告格式,可用chapter,part,默认封面、目录自动分页。
%ctexbook:书籍格式,同上,奇偶页左右边距不同。
%ctexart:文章格式,不可用chapter,part,封面,目录同正文同一页。

%===================标题信息
\def \title {市场综合管理信息服务平台\\\, \\技术方案}
\def \author {作者}
\def \department {部门名称}
\def \custom	{信阳金牛大别山农产品现代物流中心}
\def \company {郑州闪创网络科技有限公司}
\def \Ecompany {Zhengzhou Lighting Creative Technology Co.,Ltd}

%===================调用宏集
\usepackage{lettrine}				%首字下沉

\usepackage{graphicx}				

\usepackage{geometry}		

\usepackage{multirow}				%表格纵向合并

\usepackage{makeidx}				%索引
\makeindex							%启动索引功能

\usepackage{array}

\usepackage{hyperref}				%引用书签

\usepackage{longtable}			%跨页表单

%===================文档从这里开始
\geometry{left=3cm,right=3cm,top=3cm,bottom=3cm}
\begin{document}	

%===================标题部分
\thispagestyle{empty}
\begin{center}
	\parbox[t][3cm][c]{\textwidth}{\Large
	\begin{center} {{\custom}} \end{center} }
	
	\parbox[t][8cm][c]{\textwidth}{{\fontsize{32pt}{20pt}{
	\begin{center} {\textbf{\title}}\end{center} }}}
	
	\parbox[t][8cm][t]{\textwidth}{
	\begin{center}  {\large{(第二期)}}\end{center} }

	\parbox[b][2cm][c]{\textwidth}{ {\large
	\begin{center}
	\company \\
	\Ecompany 
	\end{center}}}
	
	\parbox[b][2cm][b]{\textwidth}{
	\begin{center} {\large\textbf{\number\year \,·\,\number\month}} \end{center} }
\end{center}

%===================目录部分
\newpage
\setcounter{page}{1}
\pagenumbering{Roman}				%页码以罗马数字计数
\tableofcontents					%生成目录

%===================正文部分
%===================章节
\chapter{项目总体功能概述及建设总路线图}
%\setcounter{page}{1}				%重置页码(注意位置,要在\chapter行下边
\pagenumbering{arabic}			%页码以阿拉伯数字表示
\lettrine[lines=2]{本}\, 方案在原架构方案基础上,结合金牛大别山农产品现代物流中心实际需求紧迫度及建设周期,提出整体平台的功能模块划分,各期建设规划。市场综合管理信息服务平台计划共分六期建设,依次是\textbf{市场及商户管理模块},\textbf{仓储管理及监控模块},\textbf{市场交易信息收集模块},\textbf{市场财务管理系统},\textbf{市场线上B2B商城},\textbf{冷链物流管理模块}。\par

\begin{description}
\item[市场及商户管理模块]主要完成市场的商铺管理,常驻商户信息管理;实现商户线上缴纳租金,费用自动统计及定期生成报表;市场内部通知公告。\par
\item[仓储管理及监控模块]主要完成市场纯人工平库的数字化货物管理,并将现阶段使用中的自动、半自动化仓库的数据统一进行抽出汇总,在此基础上完成后台对所有类别仓库的信息查看,定向查询。在仓库内设置多点位视频探头及温控探头,并通过数据接口实现管理后台对于仓库的实时监控,实现异常温度自动报警。\par
\item[市场交易信息收集模块]主要完成市场内交易货物的采样及统计功能。本模块通过商户主动货物申报,车辆扫牌识别,地磅等软硬件联动,并结合仓库出入库货物信息,实现市场内车辆管理、吞吐货物量估算、客流量统计。
\item[市场财务管理模块]主要完成市场财务相关管理,包括日常记账,账户管理,固定资产管理,期末结转,损益结算,财务报表,报税,银企直联,日常报销等。
\item[市场线上B2B商城]主要完成市场线上B2B商城建设,通过供、需两方面建立市场上下游供应链,建立线上商品统筹配送,并对接物流企业完成自动发货,逐步开拓线上零售市场。
\item[冷链物流管理模块]主要完成市场供应链建设过程中的运输,仓储监控功能,配合市场线上商城以及作为市场后续发展方向的冷链物流业务。
\end{description}

本项目由郑州闪创网络科技有限公司提供全程项目技术研发,软硬件接口对接\footnote{部分已有软硬件设备接口需要市场方提供}以及项目的\hyperref[arrange]{部署实施},\hyperref[arrange]{员工培训},技术维护。

本项目第一期已经完成建设的内容表~\ref{1st_s}所示:

\begin{table}[htbp]
\begin{tabular*}{\hsize}{p{3cm}<{\centering}|p{3cm}<{\centering}|p{8cm}}
\hline
模块划分					&	平台				&	\multicolumn{1}{c}{主要功能}		\\
\hline
市场前端(第一期)		&	PC网站,公众号		&	市场信息,商户登录,商户信息,缴纳租金,系统通知		\\
\hline
市场管理后台(第一期)	&	PC网站				&	商户信息,商铺信息,租金管理,发布通知,缴费信息统计及报表		\\	
\hline
\end{tabular*}
\caption{项目一期建设功能模块}
\label{function}
\end{table}

本方案着重介绍第二期建设及实施路线图。第二期建设内容如表~\ref{function}所示:

\begin{table}[htbp]
\begin{tabular*}{\hsize}{p{3cm}<{\centering}|p{3cm}<{\centering}|p{8cm}}
\hline
模块划分					&	平台				&	\multicolumn{1}{c}{主要功能}		\\
\hline
市场前端(第一期)		&	PC网站,公众号		&	市场信息,商户登录,商户信息,缴纳租金,系统通知		\\
\hline
市场管理后台(第一期)	&	PC网站				&	商户信息,商铺信息,租金管理,发布通知,缴费信息统计及报表		\\	
\hline
\end{tabular*}
\caption{项目一期建设功能模块}
\label{function}
\end{table}

通过项目第二期建设,主要解决市场仓储管理的各项问题:
\begin{itemize}
\item 一是解决现阶段纯人工平库监管不足,缺乏数据统计的问题;
\item 二是解决自动化库、半自动化库同人工库的数据整合问题;
\item 三是为商户提供线上查看仓储,申请仓储服务的功能。
\item 四是使市场管理者可以线上监控仓库实时情况,线上查看仓库使用情况及吞吐量。
\end{itemize}

%===================章节
\chapter{仓储管理系统业务逻辑}
\lettrine[lines=2]{市}\, 
\section{}

%===================章节
\chapter{仓储管理系统业务前端}
\section{}

%===================章节 
\chapter{市场PC网站功能(第二期)}
\section{仓储业务介绍}
\section{我的仓库}							%查库,费用缴纳
\section{货物托管申请}

%===================章节
\chapter{市场公众号建设(第二期)}
\section{仓储业务介绍}
\section{我的仓库}
\section{货物托管申请}

%===================章节
\chapter{市场管理后台(第二期)}
\section{仓库监控}							%温度监控,温度预警,库门监控
\section{库存管理}							%库存查询,
\section{历史记录}							%出入库记录
\section{数据报表}							%新增定期在库报表,吞吐量报表,商户使用量报表。

%===================章节
\chapter{项目预期进度}

\begin{table}[htbp]
\begin{tabular*}{\hsize}{p{2.5cm}<{\centering}@{-}p{2.5cm}<{\centering}|p{7cm}|p{3cm}}
\hline
\multicolumn{2}{c|}{时间节点}		&	进度内容				&	备注	\\
\hline
4月27日		&	4月30日		&	项目立案,需求整理,原型设计		&			\\
5月6日		&	5月10日		&	UI设计								&			\\
5月11日		&	5月29日		&	技术开发							&			\\
5月30日		&	5月31日		&	项目测试,试运行					&			\\
\multicolumn{2}{c|}{6月初}			&	项目交付				&			\\		
\hline
\end{tabular*}
\caption{项目预期建设进度}
\label{schedule}
\end{table}

%===================章节
\chapter{项目部署实施}
\label{arrange}

%===================附录部分
\appendix							%附录部分以\appendix开始,后续同正文样式

\chapter{第一期项目建设所需各类数据表单及接口}
\chapter{项目第一期报价单}

%===================参考文献部分
%\begin{thebibliography}{*}		%*缺省
%\bibitem{bib1} sth.,sth.,sth.,sth.
%\end{thebibliography}

%===================索引部分
%\printindex						%输出索引,输出前需先进行编译“MakeIndex”

\end{document}